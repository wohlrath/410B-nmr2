\section*{Teoretická část}
Při metodě "inversion recovery" (IR) použijeme $\pi$-pulz ke změně znaménka podélné složky magnetizace. 
Tu pak necháme relaxovat a po čase $t_w$ použijeme $\pi/2$ pulz k otočení magnetizace do příčného směru.
Poté budeme pozorovat FID signál, jehož amplituda bude \cite{skripta}
\begin{equation} \label{e:IR}
A_{FID}(t_w)= A_0 | 1 - 2 \exp(-t_w/T_1)  | \,,
\end{equation}
kde $A_0$ je konstanta a $T_1$ je spin-mřížková relaxační doba.

Spin-spinovou relaxační dobu $T_2$ změříme podle závislosti amplitudy signálu spinového echa (SE) na odstupu pulzů $t_w$, která má podle \cite{skripta} tvar
\begin{equation} \label{e:SE}
A_{SE}(t_w)=A_1 \exp(-2t_w/T_1) \exp(-kt_w^3) \,,
\end{equation}
kde $A_1$ a $k$ jsou konstanty.

U málo viskózních kapalin platí přibližně \cite{skripta}
\begin{equation}
T_1 \approxeq T_2 \,.
\end{equation}

Paramagnetické příměsy v diamagnetických kapalinách výrazně zkracují obě relaxační doby $T_1$ i $T_2$ \cite{skripta}. Podle \cite{skripta} platí, že relaxační rychlost je přímo úměrná molární objemové koncentraci iontů Cu$^{2+}$
\begin{equation} \label{e:C}
\frac{1}{T_1} \propto c_{\text{Cu}^{2+}} \,.
\end{equation}